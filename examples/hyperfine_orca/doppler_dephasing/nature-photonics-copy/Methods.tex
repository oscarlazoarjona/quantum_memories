\section*{Methods}
\subsection*{Experiment}
We use the Cs D2 line at 852 nm for our signal field, with $6S_{1/2}(F=4)$ as the ground state $|g\rangle$ and the $6P_{3/2}(F=3,4,5)$ manifold as the ORCA intermediate state $|e\rangle$. A strong 917 nm control field couples this signal to the storage state $|s\rangle$, i.e. the $6D_{5/2}(F=2,3,4,5,6)$ manifold. This configuration can be reasonably treated as a three-level system under broadband excitation \cite{Huber2011}. We detune both fields by 6 GHz from the intermediate state towards the ground state, enabling good coupling with negligible ($<2\%$) linear absorption.

The generation of heralded single photons is achieved using type-II parametric down-conversion (PDC) in a periodically poled potassium titanyl phosphate waveguide. The source generates THz-bandwidth pairs of signal and idler photons, both of which are consequently filtered down to $\sim1$ GHz bandwidth centred at the signal frequency using a series of Fabry-Per\'ot etalons and grating filters \cite{Michelberger2015}. We send the PDC idler to a single-photon silicon avalanche photodiode (APD) and the signal towards the memory. Our heralding efficiency before the memory is $\eta_\textrm{herald}\approx5\,\%$. The source is pumped at a rate of 80 MHz from a frequency doubled actively mode-locked titanium sapphire laser, with a $\sim0.8\%$ chance of producing a photon pair of the correct bandwidth per pump pulse.

The control field is derived from a second mode-locked titanium sapphire laser, locked-to-clock with the PDC pump. The laser bandwidth is $\sim0.9$ GHz, corresponding to a pulse duration of $\sim500$ ps. In order to investigate storage times $<12.5$ ns, i.e. smaller than the time between consecutive pulses from the laser, we use an unbalanced Mach-Zehnder interferometer with a variable delay in one arm to split the control pulse train into two: read-in and read-out, and delay them with respect to each other. The read-in and read-out control pulse energies are 0.21(1) and 0.97(1) nJ, respectively.

We combine the signal and counter-propagating control fields on a dichroic mirror. Both beams are focused down to a $\sim300\;\mu$m waist and temporally overlapped inside a 72 mm -long uncoated glass Cs-133 vapour cell heated to $\sim 91^o$C.

After the signal field leaves the memory it is sent into a Hanbury-Brown-Twiss detection setup, composed of a balanced beamsplitter and two fibre-coupled single-photon silicon APDs connected to a time-to-digital converter (same as the idler detector), allowing us to reconstruct the quantum photon number statistics of the stored/recalled signal fields.

\subsection*{Photon statistics}
By evaluating the $g^{(1,1)}$ cross-correlation function between the signal and idler (herald) pulses, we obtain a measure for the strength of the correlations between them. $g^{(1,1)}$ is defined as $p_\mathrm{si}/p_\mathrm{s}p_\mathrm{i}$, where $p_\mathrm{si}$ is the probability for a signal-idler coincidence click, and $p_\mathrm{s(i)}$ is the signal (idler) click probability. Values of $g^{(1,1)}>2$ signify non-classical correlations. To calculate $g^{(1,1)}$ from the measurements, we use

\begin{equation}
g^{(1,1)}=\frac{R_\mathrm{s,i}}{R_\mathrm{s}R_\mathrm{i}}R_\mathrm{T},
\end{equation}
where $R_\mathrm{s,i}$ is the sum of $\mathrm{D_{i}}$-$\mathrm{D_{s1}}$ and $\mathrm{D_{i}}$-$\mathrm{D_{s2}}$ coincidences, $R_\mathrm{T}$  is the total number of trigger events during the whole measurement time, $R_\mathrm{s}$ is the sum of $\mathrm{D_{s1}}$ and $\mathrm{D_{s2}}$ clicks, and $R_\mathrm{i}$ is the number of $\mathrm{D_{i}}$ clicks.

The heralded auto-correlation is defined as $g^{(2)}_\mathrm{h}=p_\mathrm{(s1,s2|i)}/p_\mathrm{(s1|i)}p_\mathrm{(s2|i)}$. Here, $p_\mathrm{(s1,s2|i)}$ is the conditional probability of detecting a coincidence between $\mathrm{D_{s1}}$ and $\mathrm{D_{s2}}$ given a click in $\mathrm{D_{i}}$, and $p_\mathrm{(s1|i)}$ ($p_\mathrm{(s2|i)}$) is the probability to detect a click in $\mathrm{D_{s1}}$ ($\mathrm{D_{s2}}$) given a click in $\mathrm{D_{i}}$. Any $g^{(2)}_\mathrm{h}<1$ verifies non-classical photon-number statistics. We evaluate $g^{(2)}_\mathrm{h}$ using

\begin{equation}
g^{(2)}_\mathrm{h}=\frac{R_\mathrm{trip}}{R_\mathrm{s1,i}R_\mathrm{s2,i}}R_\mathrm{i},
\end{equation}
where $R_\mathrm{trip}$ is the number of triple coincidences between $\mathrm{D_{i}}$, $\mathrm{D_{s1}}$, and $\mathrm{D_{s2}}$, $R_\mathrm{i}$ is the number of idler clicks, and $R_\mathrm{s1(2),i}$ is the number of $\mathrm{D_{i}}$-$\mathrm{D_{s1}}$ ($\mathrm{D_{i}}$-$\mathrm{D_{s2}}$) coincidences. More details can be found in the Supplementary Information.

\subsection*{Theoretical modeling}

An ab-initio model using Maxwell-Bloch equations is used to characterize the memory performance. We obtain equations for the time evolution of  velocity class ensembles $\hat{\rho}(v)$ coupled with signal and control fields. We consider 12 atomic states corresponding to the hyperfine levels of the $6S_{1/2}, 6P_{3/2}, 6D_{5/2}$ multiplets. The fields are coupled to the atomic states through a master equation with a Hamiltonian under electric dipole and rotating wave approximations, and in an appropriately chosen rotating frame, as well as Lindblad terms accounting for spontaneous decay5. The control field is taken to be strong enough to propagate without dispersion from the atomic vapour. The signal field is coupled to the total density matrix $\hat{\rho}=\int \mathrm{d}v g(v) \hat{\rho}(v)$ (where $g(v)$ is a Maxwell-Boltzmann velocity distribution) through the source term of a wave equation under the slowly-varying envelope approximation.

We numerically solve these equations using experimental parameters with only electric dipole matrix elements as free parameters. Once a good fit is found for the memory efficiencies as a function of control pulse power, the same parameters are used throughout this work.
